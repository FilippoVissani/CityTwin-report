\documentclass[12pt,a4paper,openright,twoside]{book}
\usepackage[utf8]{inputenc}

\newcommand{\thesislang}{italian} % decommentare in caso di tesi in italiano
%\newcommand{\thesislang}{english} % commentare in caso di tesi in italiano
\usepackage{thesis-style}

\begin{document}
	
\frontmatter

% ! TeX root = thesis-main.tex
\title{Title}
\author{Candidate Name Here}
\date{\today}

\newgeometry{margin=0.8in}
\begin{titlepage}
	\begin{center}
		% \vspace*{0.2cm}
		
		\large
		\textbf{ALMA MATER STUDIORUM -- UNIVERSITÀ DI BOLOGNA \\ CAMPUS DI CESENA}
		\\
		\noindent\hrulefill
		\vspace{0.4cm}
		
		\Large
		Scuola di Ingegneria e Architettura \\
		Corso di Laurea Magistrale in Ingegneria e Scienze Informatiche
		
		\Huge
		\vspace{4cm}
		\textbf{
			CityTwin
		}
		
		\large
		\vspace{1cm}
		Digital twin nel contesto delle smart city
		\\
		\vspace{5.5cm}
		\begin{minipage}[t]{0.64\textwidth}
			\begin{flushleft} 
				\textit{Filippo Vissani}
				\\
				filippo.vissani@studio.unibo.it
				\\
				1026702
				\\
				\vspace{0.4cm}
				\textit{Eddie Barzi}
				\\
				eddie.barzi@studio.unibo.it
				\\
    				1045197
			\end{flushleft}
		\end{minipage}
		
		\vfill
		\noindent\hrulefill
		\vspace{0.3cm}
		\Large
		\\
		Anno Accademico 2022-2023
	\end{center}
\end{titlepage}
\restoregeometry


\begin{abstract}
    L'obiettivo di CityTwin è quello di realizzare la simulazione di un sistema di digital twin nel contesto della smart city. In particolare, si vuole realizzare un sistema che sia in grado di catturare e rappresentare in formato digitale il comportamento delle varie entità presenti all'interno della città. Questo può portare ad una serie di benefici, alcuni dei quali vengono elencati di seguito:
    \begin{itemize}
        \item Rilevazione di possibili problematiche e intervento tempestivo/automatizzato.
        \item Riduzione del consumo energetico.
        \item Rilevazione della qualità dell'aria e dell'acqua.
        \item Analisi dell'inquinamento acustico.
        \item Ottimizzazione della mobilità urbana.
    \end{itemize}
\end{abstract}

%----------------------------------------------------------------------------------------
\tableofcontents   
% \listoffigures     % (optional) comment if empty
% \lstlistoflistings % (optional) comment if empty
%----------------------------------------------------------------------------------------

\mainmatter

%----------------------------------------------------------------------------------------
\chapter{Obiettivi del Progetto}
\label{chap:goals}
%----------------------------------------------------------------------------------------

\section{Requisiti Funzionali}

All'interno della simulazione saranno presenti due tipologie di nodi: focal ed entity.

\begin{enumerate}
    \item I nodi focal sono la struttura portante dell'intero sistema:
    \begin{enumerate}
        \item Scambiano informazioni nodi entity.
        \item Rilevano i malfunzionamenti dei nodi entity.
        \item Si occupano di salvare in modo persistente le informazioni.
    \end{enumerate}
    \item I nodi entity rappresentano astrazioni di sensori, attuatori o entità più complesse.
    \item I nodi entity interagiscono direttamente solo con i nodi focal per ottenere o comunicare informazioni.
    \item Deve essere possibile aggiungere o rimuovere nodi entity in tempo reale.
    \item Il sistema deve essere modulare:
    \begin{enumerate}
        \item I nodi focal non devono conoscere a prescindere le tipologie dei nodi entity.
    \end{enumerate}
    \item Deve essere presente una GUI che presenti le seguenti informazioni:
    \begin{enumerate}
        \item Lo stato attuale dei nodi focal ed entity del sistema.
        \item Lo storico dei dati registrato dai nodi focal.
    \end{enumerate}
\end{enumerate}

\section{Tecnologie Utilizzate}

\begin{itemize}
    \item Scala + SBT
    \item Akka
    \item MongoDB
    \item Docker
\end{itemize}

%----------------------------------------------------------------------------------------
\chapter{Piano di Lavoro Previsto}
\label{chap:expectedWorkPlan}
%----------------------------------------------------------------------------------------

%----------------------------------------------------------------------------------------
% BIBLIOGRAPHY
%----------------------------------------------------------------------------------------

%\nocite{*} % uncomment this to show all the reference in the .bib file
%\bibliographystyle{plain}
%\bibliography{bibliography}


\end{document}