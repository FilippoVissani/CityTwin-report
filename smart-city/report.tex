\documentclass[12pt]{article}

\usepackage[utf8]{inputenc}
%\usepackage[T1]{fontenc}

\usepackage{geometry}
\geometry{a4paper}
\usepackage{graphicx}
\usepackage{float}
\usepackage[italian]{babel}

\linespread{1.2}
\setlength{\parindent}{0pt}

\begin{document}

\renewcommand{\labelenumii}{\arabic{enumi}.\arabic{enumii}}
\renewcommand{\labelenumiii}{\arabic{enumi}.\arabic{enumii}.\arabic{enumiii}}
\renewcommand{\labelenumiv}{\arabic{enumi}.\arabic{enumii}.\arabic{enumiii}.\arabic{enumiv}}
%----------------------------------------------------------------------------------------
%	TITOLO
%----------------------------------------------------------------------------------------

\begin{titlepage}

\newcommand{\HRule}{\rule{\linewidth}{0.5mm}}

\center

\textsc{\Large Relazione di progetto di "Smart City e Tecnologie Mobili"}\\[0.5cm]

\HRule \\[0.4cm]
{ \huge \bfseries CityTwin}\\[0.4cm]
\HRule \\[1.5cm]

\vfill

\begin{flushleft}
\emph{Numero del gruppo: 128}\\[1cm]
\emph{Componenti del gruppo: Eddie Barzi, Filippo Vissani}\\[3cm]
\end{flushleft}

\end{titlepage}

%----------------------------------------------------------------------------------------
%	INDICE
%----------------------------------------------------------------------------------------

\tableofcontents

\newpage

%----------------------------------------------------------------------------------------
%	INTRODUZIONE
%----------------------------------------------------------------------------------------

\section{Introduzione}

Il progetto CityTwin si propone di realizzare la simulazione di un sistema di digital twin nel contesto della smart city. In particolare, si vuole realizzare un sistema che sia in grado di catturare e rappresentare in formato digitale il comportamento delle varie entità presenti all'interno della città. Questo può portare ad una serie di benefici, alcuni dei quali vengono elencati di seguito:

\begin{itemize}
    \item Rilevazione di possibili problematiche e intervento tempestivo/automatizzato.
    \item Riduzione del consumo energetico.
    \item Rilevazione della qualità dell'aria e dell'acqua.
    \item Analisi dell'inquinamento acustico.
    \item Ottimizzazione della mobilità urbana.
\end{itemize}

La simulazione sarà composta da due tipologie di nodi: i nodi Mainstay, che rappresentano la struttura portante del sistema, e i nodi Resource, che rappresentano astrazioni di sensori, attuatori o entità più complesse.

I nodi Mainstay si occupano di scambiare informazioni con i nodi Resource, rilevare eventuali malfunzionamenti e salvare in modo persistente le informazioni rilevate dai nodi Resource. I nodi Mainstay devono essere sempre sincronizzati tra loro, in modo da poter garantire la coerenza dei dati.

I nodi Resource, invece, si occupano di rilevare informazioni e comunicarle ai nodi Mainstay nel caso in cui vengono considerati come sensori. Nel caso in cui i nodi Resource rappresentino attuatori, invece, si occupano di ricevere informazioni dai nodi Mainstay e agire di conseguenza.

Per la memorizzazione delle dello stato dei nodi viene disposto un servizio apposito di persistenza dei dati. Tale servizio viene utilizzato sia dai nodi Mainstay che da altri clienti, come ad esempio il pannello di controllo.

L'utente potrà visualizzare lo stato attuale del sistema, lo storico dei dati ed eventuali statistiche, nonché interagire con il sistema tramite GUI, ad esempio per intervenire dopo la rilevazione di un incendio.

Per la realizzazione del progetto verranno utilizzate le seguenti tecnologie:

Akka è un toolkit open source e runtime che semplifica la costruzione di applicazioni concorrenti e distribuite sulla JVM. Questo toolkit supporta diversi modelli di programmazione per la concorrenza, ma enfatizza la concorrenza basata su attori.

I componenti principali del progetto, quali Mainstay e Resource, verranno modellati sulla base del paradigma ad attori e verranno realizzati utilizzando Akka in combinazione con il linguaggio Scala 3.

Per quanto riguarda il servizio di persistenza dei dati, utile per visualizzare le statistiche, verrà utilizzato MongoDB, un database non relazionale orientato ai documenti. Questo database verrà utilizzato per salvare in modo persistente i dati rilevati dai nodi Mainstay.
Il servizio di persistenza verrà realizzato come modulo separato, per questo motivo si sceglie di implementare un layer scritto in JavaScript, che esponga delle API utilizzabili sia dai nodi Mainstay che da altri clienti.

Sulla base dei linguaggi scelti, verranno adottati Simple Build Tool (SBT) per Scala 3 e Node Package Manager (NPM) per JavaScript per la gestione delle dipendenze e la compilazione del codice.

Per semplificare l'avvio del sistema distribuito e garantire la scalabilità verrà utilizzato Docker, un progetto open source che automatizza il deployment di applicazioni all'interno di contenitori software.

\newpage

%----------------------------------------------------------------------------------------
%	STATO DELL'ARTE
%----------------------------------------------------------------------------------------

\section{Stato dell'arte}



\newpage


%----------------------------------------------------------------------------------------
%	ANALISI DEI REQUISITI
%----------------------------------------------------------------------------------------

\section{Analisi dei requisiti}

Di seguito viene riportato l'elenco dettagliato dei requisiti del sistema.

\begin{enumerate}
    \item I nodi Mainstay sono la struttura portante dell'intero sistema:
    \begin{enumerate}
        \item Ricevono informazioni dai nodi Resource che modellano sensori.
        \item Comunicano informazioni ai nodi Resource che modellano attuatori.
        \item Rilevano i malfunzionamenti dei nodi Resource.
        \item Rilevano i malfunzionamenti degli altri nodi Mainstay.
        \item Si occupano di salvare le informazioni contattando il servizio di persistenza.
        \item Non devono conoscere a prescindere le tipologie dei nodi Resource.
        \item Devono disporre di una struttura dati distribuita che permetta di memorizzare le informazioni rilevanti:
        \begin{enumerate}
            \item La struttura dati deve essere sincronizzata tra i nodi Mainstay.
            \item La struttura dati deve mantenere la consistenza dei dati nel tempo.
        \end{enumerate}
    \end{enumerate}
    \item I nodi Resource:
    \begin{enumerate}
        \item Rappresentano astrazioni di:
        \begin{enumerate}
            \item Sensori.
            \item Attuatori.
            \item Entità più complesse, come stazioni di controllo, che possono anche impiegare interfacce grafiche.
        \end{enumerate}
        \item Fanno riferimento ad uno dei nodi Mainstay per ottenere o comunicare informazioni.
        \item Possono essere aggiunti o rimossi in tempo reale.
    \end{enumerate}
    \item Deve essere presente una GUI (distinta da quelle definite al punto 2.1.3) con le seguenti funzionalità:
    \begin{enumerate}
        \item Deve mostrare lo stato attuale dei nodi Mainstay e Resource del sistema.
        \item Deve mostrare la posizione delle risorse nella città.
        \item Deve presentare alcune statistiche interessanti sulla base dei dati rilevati dal servizio di persistenza.
    \end{enumerate}
    \item Deve essere presente un servizio di persistenza dei dati che permetta di salvare in modo persistente:
    \begin{enumerate}
        \item Lo stato dei nodi Mainstay.
        \item Lo stato dei nodi Resource.
    \end{enumerate}
    \item Il malfunzionamento di un nodo non deve compromettere il funzionamento del sistema.
    \item Il sistema deve essere realizzato in moduli separati, in modo da poter essere facilmente estendibile.
    \item Deve poter essere possibile introdurre nuovi moduli senza dover modificare i componenti esistenti.
    \item Vengono previsti i seguenti moduli per i nodi Resource:
    \begin{enumerate}
        \item Sensore per la rilevazione di piogge acide.
        \item Sensore per l'analisi della qualità dell'aria.
        \item Sensore per la rilevazione dell'inquinamento acustico.
        \item Modulo per la rilevazione del livello dell'acqua di un fiume e possibilità di intervento:
        \begin{enumerate}
            \item Sensore per la rilevazione degli allagamenti.
            \item Pannello di controllo per l'intervento in caso di allagamento.
        \end{enumerate}
    \end{enumerate}
\end{enumerate}

\newpage


%----------------------------------------------------------------------------------------
%	PROGETTAZIONE
%----------------------------------------------------------------------------------------

\section{Progettazione}

L'architettura del sistema, presentata in Figura \ref{fig:core-component-diagram}, è organizzata intorno a componenti interconnessi che consentono il monitoraggio e la gestione delle entità presenti all'interno della città. Ogni componente svolge ruoli specifici all'interno del sistema e interagisce attraverso interfacce ben definite.

\subsection{Suddivisione dei Componenti}

\subsubsection{Core}
Il componente centrale del sistema è il \textit{Core}, responsabile della gestione dell'elaborazione dati, del coordinamento delle attività e della comunicazione tra gli altri componenti. Il Core funge da mediatore tra le varie risorse presenti all'interno della città, ricevendo dati dai sensori e inviandoli al servizio di persistenza. Inoltre.

\subsubsection{Componenti Monitor}
I diversi monitor ambientali, come l'\textit{Acid Rain Monitor}, l'\textit{Air Quality Monitor}, il \textit{Noise Pollution Monitor} e il \textit{River Monitor}, costituiscono le fonti primarie di dati. Ciascun monitor raccoglie misurazioni specifiche e invia queste informazioni al \textit{Core} tramite l'interfaccia \textit{Resource}. Questa comunicazione consente al \textit{Core} di elaborare i dati e fornire una visione completa delle condizioni ambientali.

\subsubsection{Control Panel}
Il \textit{Control Panel} è l'interfaccia utente principale attraverso cui gli utenti interagiscono con il sistema. Comunica con il \textit{Core} per ottenere dati sullo stato ambientale e sul sistema nel suo complesso. Inoltre, il pannello di controllo richiede al servizio di persistenza lo storico dei dati in modo da poterli elaborare per ottenere delle statistiche.

\subsubsection{Persistence Service}
Il \textit{Persistence Service} gestisce la persistenza dei dati nel sistema. Comunica con il \textit{Core} attraverso le interfacce \textit{Post Resource} e \textit{Post Mainstay} per ricevere e salvare i dati relativi alle risorse. Analogamente, le interfacce \textit{Get Mainstay} e \textit{Get Resource }consentono di richiedere i dati al servizio di persistenza.

\begin{figure}[H]
\caption{Diagramma dei componenti del sistema e delle loro dipendenze.}
    \includegraphics[width=\textwidth]{../assets/images/core-component-diagram.png}
    \label{fig:core-component-diagram}
\end{figure}

In Figura \ref{fig:nodes-component-diagram} viene presentata l'architettura dei componenti in esecuzione e dei protocolli di rete utilizzati. Il sistema è costituito da un insieme di nodi \textit{Mainstay} e \textit{Resource} che comunicano tra loro attraverso il protocollo \textit{Akka}. Inoltre, i nodi \textit{Mainstay} comunicano con il servizio di persistenza attraverso il protocollo \textit{HTTP}.
I nodi che comunicano utilizzando il protocollo \textit{Akka} sono \textit{Actor System} organizzati in un cluster, in modo da poter garantire la scalabilità del sistema.

\begin{figure}[H]
    \caption{Diagramma dei componenti in esecuzione e dei protocolli di rete utilizzati.}
    \includegraphics[width=\textwidth]{../assets/images/nodes-component-diagram.png}
    \label{fig:nodes-component-diagram}
\end{figure}

\subsection{Architettura del Modulo Core}

L'architettura del modulo \textit{Core} (Figura \ref{fig:core-class-diagram}) è costituita da diversi attori, ognuno dei quali svolge un ruolo specifico nel processo di acquisizione, gestione e comunicazione dei dati relativi alle risorse.

\textit{Resource Actor} è responsabile della gestione della risorsa, indipendentemente dal fatto che essa sia un sensore o un attuatore. Comunica con il \textit{Mainstay Actor} per ottenere o mandare lo stato delle risorse. È in grado di ricevere comandi e cambiamenti relativi alle risorse.

Il \textit{Mainstay Actor} si occupa di tenere in piedi l'intero sistema distribuito. Esso scambia lo stato delle risorse con il \textit{Resource Actor} e comunica con il \textit{Persistence Service Driver Actor} per salvare i dati nel servizio di persistenza. Inoltre, il \textit{Mainstay Actor} è responsabile della sincronizzazione dei nodi \textit{Mainstay}.

Il \textit{Nodes Observer Actor} monitora gli stati dei nodi all'interno del sistema e aggiorna il \textit{Mainstay Actor} sulla base del cambiamento di stato dei nodi.

L'attore \textit{Persistence Service Driver} è responsabile della gestione delle operazioni di persistenza dei dati. Comunica con il \textit{Mainstay Actor} per pubblicare nuovi dati al servizio di persistenza.

L'architettura prevede interazioni chiare e ben definite tra gli attori attraverso l'utilizzo di comandi specifici.

Il \textit{Resource Actor} comunica con il \textit{Mainstay Actor} utilizzando comandi come \textit{AskResourcesState}, \textit{AskAllResourcesState} e \textit{UpdateResources}. Queste interazioni consentono al \textit{Resource Actor} di ottenere informazioni sullo stato delle risorse e di aggiornare lo stato stesso.

Il \textit{Mainstay Actor} comunica con il \textit{Persistence Service Driver Actor} utilizzando comandi come \textit{PostMainstay} e \textit{PostResource}. Queste interazioni consentono al \textit{Mainstay Actor} di inviare nuovi dati al servizio di persistenza.

\begin{figure}[H]
    \caption{Diagramma delle classi del modulo \textit{Core}.}
    \includegraphics[width=\textwidth]{../assets/images/core-class-diagram.png}
    \label{fig:core-class-diagram}
\end{figure}

\subsection{Definizione delle Interazioni dei Componenti}

Il comportamento generale del sistema viene definito sulla base di una serie di interazioni tra i componenti. In particolare, le interazioni possono essere raggruppate per definire un determinato aspetto di tale comportamento.

\subsubsection{Aggiornamento Sullo Stato dei Nodi}

Lo scambio di messaggi tra \textit{Nodes Observer Actor} e \textit{Mainstay Actor} (Figura \ref{fig:core-nodes-state-sequence-diagram}) è volto a mantenere aggiornato quest'ultimo sullo stato generale dei nodi del sistema. Nel momento in cui il \textit{Mainstay Actor} riceve un aggiornamento sui nodi, esso aggiorna il proprio stato e lo comunica al \textit{Persistence Service Driver Actor}, che si occupa di rendere persistenti i dati tramite il servizio di persistenza.

\begin{figure}[H]
    \caption{Diagramma di sequenza per l'aggiornamento sullo stato dei nodi del sistema.}
    \includegraphics[width=\textwidth]{../assets/images/core-nodes-state-sequence-diagram.png}
    \label{fig:core-nodes-state-sequence-diagram}
\end{figure}

\subsubsection{Aggiornamento Sullo Stato delle Risorse}

Lo scambio di messaggi tra \textit{Resource Actor} e \textit{Mainstay Actor} (Figura \ref{fig:core-resource-state-exchange-sequence-diagram}) è utile a:
\begin{itemize}
    \item Ottenere lo stato delle altre risorse e comunicare il proprio nel caso un cui il nodo \textit{Resource} modella un attuatore.
    \item Comunicare lo stato della risorsa nel caso un cui il nodo \textit{Resource} modella un sensore.
\end{itemize}

In ogni caso, quando il \textit{Mainstay Actor} riceve un aggiornamento, lo comunica al servizio di persistenza tramite il \textit{Persistence Service Driver Actor}.

\begin{figure}[H]
    \caption{Diagramma di sequenza per lo scambio dello stato delle risorse.}
    \includegraphics[width=\textwidth]{../assets/images/core-resource-state-exchange-sequence-diagram.png}
    \label{fig:core-resource-state-exchange-sequence-diagram}
\end{figure}

\subsubsection{Sincronizzazione dei Nodi Mainstay}

Come detto precedentemente, i nodi \textit{Mainstay} sono i pilastri portanti dell'intero sistema e si occupano della gestione dei dati riguardanti le risorse e i nodi. I nodi \textit{Mainstay} devono essere sempre sincronizzati tra loro, in modo da poter garantire la coerenza dei dati. Per questo motivo, i nodi \textit{Mainstay} si sincronizzano ogni volta che una risorsa manda il suo stato ad uno di questi. In Figura \ref{fig:core-mainstays-sync-sequence-diagram} viene presentato il diagramma di sequenza per la sincronizzazione dei nodi \textit{Mainstay}.

\begin{figure}[H]
    \caption{Diagramma di sequenza per la sincronizzazione dei nodi Mainstay.}
    \includegraphics[width=\textwidth]{../assets/images/core-mainstays-sync-sequence-diagram.png}
    \label{fig:core-mainstays-sync-sequence-diagram}
\end{figure}

\subsubsection{Aggiornamento del Pannello di Controllo}

Il comportamento del \textit{Control Panel} (Figura \ref{fig:control-panel-sequence-diagram}) viene modellato come un caso particolare di risorsa che comunica con il \textit{Mainstay Actor} per ottenere lo stato delle risorse e dei nodi. Inoltre, il \textit{Control Panel} comunica con il servizio di persistenza per ottenere lo storico dei dati e calcolare le statistiche.

\begin{figure}[H]
    \caption{Diagramma di sequenza per l'aggiornamento del pannello di controllo.}
    \includegraphics[width=\textwidth]{../assets/images/control-panel-sequence-diagram.png}
    \label{fig:control-panel-sequence-diagram}
\end{figure}

\newpage

%----------------------------------------------------------------------------------------
%	IMPLEMENTAZIONE
%----------------------------------------------------------------------------------------

\section{Implementazione}\label{sec:implementazione}

\begin{figure}[H]
    \caption{Pannello di controllo: schermata della mappa della città.}
    \includegraphics[width=\textwidth]{../assets/images/control-panel-map.png}
\end{figure}

\begin{figure}[H]
    \caption{Pannello di controllo: schermata delle informazioni sullo stato dei nodi \textit{Resource} e dei nodi \textit{Mainstay}.}
    \includegraphics[width=\textwidth]{../assets/images/control-panel-info.png}
\end{figure}

\begin{figure}[H]
    \caption{Pannello di controllo: schermata delle statistiche dei nodi \textit{Resource}.}
    \includegraphics[width=\textwidth]{../assets/images/control-panel-resources-stats.png}
\end{figure}

\newpage


%----------------------------------------------------------------------------------------
%	TESTING E PERFORMANCE
%----------------------------------------------------------------------------------------

\section{Testing e performance}



\newpage


%----------------------------------------------------------------------------------------
%	ANALISI DI DEPLOYMENT SU LARGA SCALA
%----------------------------------------------------------------------------------------

\section{Analisi di deployment su larga scala}



\newpage


%----------------------------------------------------------------------------------------
%	PIANO DI LAVORO
%----------------------------------------------------------------------------------------

\section{Piano di lavoro}



\newpage


%----------------------------------------------------------------------------------------
%	CONCLUSIONI
%----------------------------------------------------------------------------------------

\section{Conclusioni}



\newpage


%----------------------------------------------------------------------------------------
%	APPENDICE
%----------------------------------------------------------------------------------------

\appendix
\addcontentsline{toc}{section}{Appendice}
\section*{Appendice}



\newpage


%----------------------------------------------------------------------------------------
%	RIFERIMENTI BIBLIOGRAFICI
%----------------------------------------------------------------------------------------
\addcontentsline{toc}{section}{Riferimenti bibliografici}
\begin{thebibliography}
    Elencare i riferimenti bibliografici citati nel testo.
\end{thebibliography}

%----------------------------------------------------------------------------------------

\end{document}