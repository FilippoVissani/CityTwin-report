\documentclass{scrartcl}
\usepackage[utf8]{inputenc}
\usepackage{hyperref}
\usepackage{url}
\usepackage{natbib}
\usepackage{graphicx}

\newcommand{\emailaddr}[1]{\href{mailto:#1}{\texttt{#1}}}

\title{\LARGE
    CityTwin
}

% Consider watching:
% https://www.youtube.com/watch?v=ihxSUsJB_14
% https://www.youtube.com/watch?v=XTFWaV55uDo

\author{
    Filippo Vissani \\ \emailaddr{filippo.vissani@studio.unibo.it}
    \and 
    Eddie Barzi \\ \emailaddr{eddie.barzi@studio.unibo.it}
}

\date{September 2023}

\begin{document}

\maketitle

\begin{abstract}
    %Up to $\sim$2000 characters briefly describing the project.

    Il progetto CityTwin si propone di realizzare la simulazione di un sistema di digital twin nel contesto della smart city. In particolare, si vuole realizzare un sistema che sia in grado di catturare e rappresentare in formato digitale il comportamento delle varie entità presenti all'interno della città. Questo può portare ad una serie di benefici, alcuni dei quali vengono elencati di seguito:

    \begin{itemize}
        \item Rilevazione di possibili problematiche con intervento tempestivo e automatizzato.
        \item Riduzione del consumo energetico.
        \item Rilevazione della qualità dell'aria e dell'acqua.
        \item Analisi dell'inquinamento acustico.
        \item Ottimizzazione della mobilità urbana.
    \end{itemize}

    La simulazione sarà composta da due tipologie di nodi: i nodi Mainstay, che rappresentano la struttura portante del sistema, e i nodi Resource, che rappresentano astrazioni di sensori, attuatori o entità più complesse.

    I nodi Mainstay si occupano di scambiare informazioni con i nodi Resource, rilevare eventuali malfunzionamenti e salvare in modo persistente le informazioni rilevate dai nodi Resource. I nodi Mainstay devono essere sempre sincronizzati tra loro, in modo da poter garantire la coerenza dei dati.

    I nodi Resource, invece, si occupano di rilevare informazioni e comunicarle ai nodi Mainstay nel caso in cui vengano considerati come sensori. Nel caso in cui i nodi Resource rappresentino attuatori, invece, si occupano di ricevere informazioni dai nodi Mainstay e agire di conseguenza.

    Per la memorizzazione delle dello stato dei nodi viene disposto un servizio apposito di persistenza dei dati. Tale servizio viene utilizzato sia dai nodi Mainstay che da altri clienti, come ad esempio il pannello di controllo.

    L'utente potrà visualizzare lo stato attuale del sistema, lo storico dei dati ed eventuali statistiche, nonché interagire con il sistema tramite GUI, ad esempio per intervenire dopo la rilevazione di un incendio.

\end{abstract}

\section{Goal/Requirements}

Detailed description of the project goals, requirements, and expected outcomes.
%
Use case Diagrams, examples, or Q/A simulations are welcome.

\subsection{Scenarios}

Informal description of the ways users are expected to interact with your project.
%
It should describe \emph{how} and \emph{why} a user should use / interact with the system.

\subsection{Self-assessment policy}

\begin{itemize}
    \item How should the \emph{quality} of the \emph{produced software} be assessed?

    \item How should the \emph{effectiveness} of the project outcomes be assessed?
\end{itemize}

\section{Requirements Analysis}

Is there any implicit requirement hidden within this project's requirements?
%
Is there any implicit hypothesis hidden within this project's requirements?
%
Are there any non-functional requirements implied by this project's requirements?

What model / paradigm / techonology is the best suited to face this project's requirements?
%
What's the abstraction gap among the available models / paradigms / techonologies and the problem to be solved?

\section{Design}

This is where the logical / abstract contribution of the project is presented.

Notice that, when describing a software project, three dimensions need to be taken into account: structure, behaviour, and interaction.

Always remember to report \textbf{why} a particular design has been chosen.
Reporting wrong design choices which has been evalued during the design phase is welcome too.

\subsection{Structure}

Which entities need to by modelled to solve the problem?
%
(UML Class diagram)

How should entities be modularised?
%
(UML Component / Package / Deployment Diagrams)

\subsection{Behaviour}

How should each entity behave?
%
(UML State diagram or Activity Diagram)

\subsection{Interaction}

How should entities interact with each other?
%
(UML Sequence Diagram)

\section{Implementation Details}

Just report interesting / non-trivial / non-obvious implementation details.

This section is expected to be short in case some documentation (e.g. Javadoc or Swagger Spec) has been produced for the software artefacts.
%
This this case, the produced documentation should be referenced here.

\section{Self-assessment / Validation}

Choose a criterion for the evaluation of the produced software and \textbf{its compliance to the requirements above}.

Pseudo-formal or formal criteria are preferred.

In case of a test-driven development, describe tests here and possibly report the amount of passing tests, the total amount of tests and, possibly, the test coverage.

\section{Deployment Instructions}

Explain here how to install and launch the produced software artefacts.
%
Assume the softaware must be installed on a totally virgin environment.
%
So, report \textbf{any} configuration step.

Gradle and Docker may be useful here to ensure the deployment and launch processes to be easy.

\section{Usage Examples}

Show how to use the produced software artefacts.

Ideally, there should be at least one example for each scenario proposed above.

\section{Conclusions}

Recap what you did

\subsection{Future Works}

Recap what you did \emph{not}

\subsection{What did we learned}

Racap what did you learned

\section*{Stylistic Notes}

Use a uniform style, especially when writing formal stuff: $X$, X, $\mathbf{X}$, $\mathcal{X}$, \texttt{X} are all different symbols possibly referring to different entities.

This is a very short paragraph.

This is a longer paragraph (notice the blank line in the code).
It composed by several sentences.
%
You're invited to use comments within \texttt{.tex} source files to separate sentences composing the same paragraph.

Paragraph should be logically atomic: a subordinate sentence from one paragraph should always refer to another sentence from within the same paragraph.

The first line of a paragraph is usually indented.
%
This is intended: it is the way \LaTeX{} lets the reader know a new paragraph is beginning.

Use the \href{https://en.wikibooks.org/wiki/LaTeX/Source_Code_Listings}{\texttt{listing}} package for inserting scripts into the \LaTeX{} source.

\nocite{*} % Includes all references from the `references.bib` file
\bibliographystyle{plain}
\bibliography{references}

\end{document}
