\documentclass[12pt,a4paper,openright,twoside]{book}
\usepackage[utf8]{inputenc}

\newcommand{\thesislang}{italian} % decommentare in caso di tesi in italiano
%\newcommand{\thesislang}{english} % commentare in caso di tesi in italiano
\usepackage{thesis-style}

\begin{document}
\renewcommand{\labelenumii}{\arabic{enumi}.\arabic{enumii}}
\renewcommand{\labelenumiii}{\arabic{enumi}.\arabic{enumii}.\arabic{enumiii}}
\renewcommand{\labelenumiv}{\arabic{enumi}.\arabic{enumii}.\arabic{enumiii}.\arabic{enumiv}}
\frontmatter

% ! TeX root = thesis-main.tex
\title{Title}
\author{Candidate Name Here}
\date{\today}

\newgeometry{margin=0.8in}
\begin{titlepage}
	\begin{center}
		% \vspace*{0.2cm}
		
		\large
		\textbf{ALMA MATER STUDIORUM -- UNIVERSITÀ DI BOLOGNA \\ CAMPUS DI CESENA}
		\\
		\noindent\hrulefill
		\vspace{0.4cm}
		
		\Large
		Scuola di Ingegneria e Architettura \\
		Corso di Laurea Magistrale in Ingegneria e Scienze Informatiche
		
		\Huge
		\vspace{4cm}
		\textbf{
			CityTwin
		}
		
		\large
		\vspace{1cm}
		Digital twin nel contesto delle smart city
		\\
		\vspace{5.5cm}
		\begin{minipage}[t]{0.64\textwidth}
			\begin{flushleft} 
				\textit{Filippo Vissani}
				\\
				filippo.vissani@studio.unibo.it
				\\
				1026702
				\\
				\vspace{0.4cm}
				\textit{Eddie Barzi}
				\\
				eddie.barzi@studio.unibo.it
				\\
    				1045197
			\end{flushleft}
		\end{minipage}
		
		\vfill
		\noindent\hrulefill
		\vspace{0.3cm}
		\Large
		\\
		Anno Accademico 2022-2023
	\end{center}
\end{titlepage}
\restoregeometry


\begin{abstract}
    L'obiettivo di CityTwin è quello di realizzare la simulazione di un sistema di digital twin nel contesto della smart city. In particolare, si vuole realizzare un sistema che sia in grado di catturare e rappresentare in formato digitale il comportamento delle varie entità presenti all'interno della città. Questo può portare ad una serie di benefici, alcuni dei quali vengono elencati di seguito:
    \begin{itemize}
        \item Rilevazione di possibili problematiche e intervento tempestivo/automatizzato.
        \item Riduzione del consumo energetico.
        \item Rilevazione della qualità dell'aria e dell'acqua.
        \item Analisi dell'inquinamento acustico.
        \item Ottimizzazione della mobilità urbana.
    \end{itemize}
\end{abstract}

%----------------------------------------------------------------------------------------
\tableofcontents   
% \listoffigures     % (optional) comment if empty
% \lstlistoflistings % (optional) comment if empty
%----------------------------------------------------------------------------------------

\mainmatter

%----------------------------------------------------------------------------------------
\chapter{Obiettivi del Progetto}
\label{chap:goals}
%----------------------------------------------------------------------------------------

\section{Requisiti Funzionali}

All'interno della simulazione saranno presenti due tipologie di nodi: Mainstay e Resource.

\begin{enumerate}
    \item I nodi Mainstay sono la struttura portante dell'intero sistema:
    \begin{enumerate}
        \item Scambiano informazioni con i nodi Resource.
        \item Rilevano i malfunzionamenti dei nodi Resource.
        \item Si occupano di salvare in modo persistente le informazioni.
        \item Non devono conoscere a prescindere le tipologie dei nodi Resource.
    \end{enumerate}
    \item I nodi Resource:
    \begin{enumerate}
        \item Rappresentano astrazioni di:
        \begin{enumerate}
            \item Sensori.
            \item Attuatori.
            \item Entità più complesse, come stazioni di controllo, che possono anche impiegare interfacce grafiche.
        \end{enumerate}
        \item Fanno riferimento ad uno dei nodi Mainstay per ottenere o comunicare informazioni.
        \item Possono essere aggiunti o rimossi in tempo reale.
    \end{enumerate}
    \item Deve essere presente una GUI (distinta da quelle definite al punto 2.1.3) con le seguenti funzionalità:
    \begin{enumerate}
        \item Deve interagire direttamente solo con i nodi Mainstay.
        \item Deve mostrare lo stato attuale dei nodi Mainstay e Resource del sistema.
        \item Deve mostrare lo storico dei dati registrato dai nodi Mainstay.
        \item Deve presentare alcune statistiche interessanti sulla base dei dati rilevati.
    \end{enumerate}
\end{enumerate}

\section{Tecnologie Utilizzate}

\begin{itemize}
    \item Scala + SBT
    \item Akka
    \item MongoDB
    \item Docker
\end{itemize}

%----------------------------------------------------------------------------------------
\chapter{Piano di Lavoro Previsto}
\label{chap:expectedWorkPlan}
%----------------------------------------------------------------------------------------

Il piano di lavoro prevede le seguenti fasi:

\begin{enumerate}
    \item Progettazione dell'architettura del sistema.
    \item Definizione delle tecniche di CI/CD da impiegare.
    \item Definizione dettagliata dei requisiti per ogni sotto-modulo del sistema.
    \item Progettazione dettagliata delle interazioni tra i sotto-moduli del sistema.
    \item Implementazione incrementale delle varie parti del sistema.
\end{enumerate}

Per quanto concerne il processo di sviluppo, verrà adottato un approccio agile, basato su sprint settimanali. Per ogni sprint verranno ripetuti i punti 3-5 definiti precedentemente.

%----------------------------------------------------------------------------------------
% BIBLIOGRAPHY
%----------------------------------------------------------------------------------------

%\nocite{*} % uncomment this to show all the reference in the .bib file
%\bibliographystyle{plain}
%\bibliography{bibliography}

\end{document}